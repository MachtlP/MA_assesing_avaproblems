\chapter*{Preface}
\addcontentsline{toc}{chapter}{Preface}
\thispagestyle{plain}

% \LaTeXe{}

"Sometimes, you just have to go with the waves." It kind of fits to my process of deciding on a thesis topic. I remember driving back home from the Atlantic, when reading an e-mail from Bernd about possible subjects. One of them even included the investigation of wave breaking

% I remember sitting at Andreas' kitchen table discussing possible investigations for the master's thesis.

% Had to implement small parts within EULAG


% This \LaTeXe{} template is based on my own dissertation. It has been extensively
% modified within the framework of the course \emph{Introduction to Scientific
% Working} which I taught at the University of Innsbruck in the winter semester
% 2009/2010 for students of the \emph{Atmospheric Sciences Master Program}. It
% does \emph{not} serve as the official thesis template of the Institute of
% Meteorology and Geophysics (IMGI), however, it follows some reasonable rules,
% such as the reference and citation guidelines of the American Meteorological
% Society. Before using this template at IMGI, make yourself familiar with the
% format guidelines of the Faculty of Geo- and Atmospheric Sciences and
% the personal preferences of your advisor. My \LaTeX{} knowledge is based on the
% guides of \citet{oeti08Aag} and \citet{kopk99Aag}. Many ideas on the content and
% structure of a science thesis are taken from the book of \citet{russ06Aag}. This
% template is distributed in the hope that it will be useful, but \emph{without
% any warranty}. I am looking forward to receiving comments for
% improvements.\footnote{\texttt{alexander.gohm [at] uibk.ac.at}}

\begin{flushright}
\textit{Michael Binder}

\textit{Innsbruck, September 2022} 
\end{flushright}
