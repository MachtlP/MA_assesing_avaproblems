\chapter{Introduction}
\label{sec:intro} 

Modern avalanche forcasting services deliver a daily report on the prevailing avalanche hazard situation. 
Up till now the assesment of the report is still mailny assest subjectevliy by experts knowledge based on 
field observations, weather and snowpack models.

Snowpack model as e.g SNOW
- SNOWPACK models 
- observations
- clusters
- NWPs
- expert assesment

Modern avalanche forcasting services 
%%% ---- 
\noindent Snowpack models as e.g. \textit{SNOWPACK} were developed more than 20 years ago but only recently they 
are used in operational avalanche warning services. Snowpack models can be driven by Automated weather stations (AWS), 
Numerical weather prediction data (NWP), initialized by observed profiles or a combination of them with the ultimate 
goal to gain informations of the evolution of the layering within the snowpack.\autocite{lehningSnowpackModelCalculations1999} \\
Avalanche forecasters are confronted with an overwhelming amount of information and either do not have the time or 
the knowledge to interpret them. Therefore there is this big need to access this tremendous data- and information 
treasure within the \textit{SNOWPACK} model more easily.
\\
\noindent At the time promising developments are ongoing to help avalanche warning services to tackle the 
interpretation of this data amount. For example \autocite{perez-guillenDatadrivenAutomatedPredictions2022}  
assigning automatically avalanche danger levels based on SNOWPACK simulations and machine-learning algorithms. 
As well as \autocite{herlaDataExplorationTool2022} averaging thousands of simulated snow profiles to a single 
one with dynamic-time-warping methods.

\noindent In 2017 the EAWS (European avalanche warning services) agreed on the so called \textit{Avalanche Problems} 
as entry point for the communication of the avalanche danger.\autocite{eawsTypicalAvalancheProblems2017} 
The assessment of the \textit{Avalanche Problems} are mostly influenced by subjectivity of the forecaster. 
Beside observed manual snow profiles and further field observations, snowpack simulations are taken into account 
to evaluate the stability within the snow pit and the prevailing problems. Until now the interpretation of the data 
is performed manually.

\noindent A recently developed algorithm for analysing the avalanche problem climatology based on snow pack 
simulations \autocite{reuterCharacterizingSnowInstability2022} will be adapted and applied on various snowpack model
 chains to obtain an objective first guess, to support the avalanche forecaster in the assessment of the prevailing 
 \textit{Avalanche Problems}. The automatically assegined problems will be compared to the assessment made by the 
 forecaster.
\\
\noindent Modern avalanche forecasting provides a forecast of the hazard description with lead times up to 31 hours. 
Still, the most direct information on snowpack stability is obtained with observed manual snow profiles and stability 
tests. This type of information is combined and interpreted with information of Numerical Weather prediction models 
(NWPs) by the forecaster. Therefore we want to investigate whether it is possible to initialize SNOWPACK by observed 
profiles and drive them with NWP-data. Furthermore we intend to apply our developed algorithm for assessing avalanche 
problems to the modeled snowpack to give the forecaster an objective first guess of an avalanche problem independent 
of weather stations.

\section{Clustering}
\noindent Modern regional avalanche hazard assessment is based on several small \textit{micro-regions} 
which are grouped during the process of the assessment in order to represent larger regions with similar 
conditions. The pre-selections and definitions of those \textit{micro-regions} is crucial for a consistent
 assessment and communication. Therefore we will investigate if modelled precipitation distribution combined 
 with cluster analysis represent an useful approach to obtain these boundaries objectively. Further we will 
 check whether the resulting outer lines of the clusters are sensitive to different prevailing flow direction 
 of synoptic weather patterns. We will focus on the area of the so called \textit{EUREGIO}, in especially the 
 Provinces of Tyrol, South Tyrol and Trentino.
\\
\\
The pre-existing outer lines of Micro Regions within the \textit{EUREGIO} (Fig. \ref{Microregions}) are 
suffering critic by local experts due to inconsistency of snow amount within a region and the prevailing 
avalanche situation. Therefore it has to be investigated if the weather patterns based on solid precipitation 
in winter months allow a better demarcation of the regions to gain an objective definition. The gained regions 
should be compared if they are in line with the subjective expert assessment. 

\begin{tcolorbox}[]
    (R1.1) Is it possible to obtain \textit{micro-regions} for avalanche forecasting based on solid precipitation 
    patterns (distribution) and are they in line with subjective expert assessment ? 
\end{tcolorbox}

\begin{tcolorbox}[]
    (R1.2) How sensitive are the outer lines of the gained \textit{micro-regions} to the prevailing wind system? 
\end{tcolorbox}

%
\section{Assessing avalanche problems - Model chains}

\subsection{AWS-model chain - Now-cast}
\textit{SNOWPACK} simulations contain an immense data treasure of the evolution of the layering within the snowpack 
and their stability. Although being developed over two decades ago they are rarely used by operational avalanche 
warning services due to its complexity and data amount. Therefore an algorithm for a direct assessment of a 
prevailing avalanche problem should be applied to help the forecaster with an first objective guess.\\
For the first steps an algorithm should be applied to the SNOWPACK-weather stations of the \textit{EUREGIO}. 
These stations are already integrated in an operating model chain, delivering direct SNOWPACK-model output 
with the evolution of the layering within the snowpack.

\begin{tcolorbox}[]
    (R2.1) Is it possible to objectively assign avalanche problems based on SNOWPACK simulations driven by 
    automated weather stations?
\end{tcolorbox}

\begin{tcolorbox}[]
    (R2.2) Does the problem assignment based on snowpack simulations behave in a fairly similar way to the
     assessment of the local avalanche forecasting service?
\end{tcolorbox}

\subsection{Observed Profiles and NWPs - Forecast } 

Despite having a good favourable position in Tyrol, not all warning services have the situation of a good 
coverage of weather stations across their forecasting area. Observed snow profiles are still the best tool to 
gain a direct information on snow pack stability tests. It should be investigated if it is possible to 
initialize \textit{SNOWPACK} by observed profiles and furthermore to drive it with the data receiving form NWPs. 
This would enable the avalanche warning service to investigate snow stability prior critical situations e.g. a
 big change of the weather pattern (as an upcoming weather front) and therefore close those data gaps. 
 Furthermore it should be evaluated for how long simulations driven by NWPs behave in a fairly similar way 
 than simulations driven by AWS. 

\begin{tcolorbox}[]
    (R3.1) Is it possible to assign avalanche problems based on snowpack simulations initialized by observed profiles and driven by NWPs?
\end{tcolorbox}

\begin{tcolorbox}[]
    (R3.2) How long are snowpack simulations and the assigned avalanche problems driven by NWPs in line with the simulations driven by AWS?
\end{tcolorbox}
