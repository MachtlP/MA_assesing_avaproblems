\chapter{Conclusion}

% Answer research questions from input!!

2D
- A reference wind speed (wind in total column - spped of transient lower boundary) can be used to generate similar wave pattern in stationary reference frame

- conclusion from 2D cases -> transfer of stationary mountain wave analysis
- valve layer! do dissipation analysis with reference wind!!

3D
- Background conditions in stratosphere can result in wave pattern observed in the case study by \cite{dornbrack_stratospheric_2022} during RF 25 of the DEEPWAVE campaign.
- Spectrum fits in terms of observed wavelenghts

- meteorlogical parameters can be observed all around jet regions 
 - what are expected wavelengths from geostrophic adjustment?

Wavelength discussion:
AIRS data:
Sensitivity is close to 100 percent for waves with wavelengths between 35 and 45 km in the vertical and less than around 500 km in the horizontal, and around 50 percent for wavelengths greater than 17 km in the vertical and less than 1000 km in the horizontal. The majority of our measured wavelengths in the results of this study fall within this 50 percent sensitivity region (as we would expect),


% Critical assumption 
% Ina (ECMWF 1km simulation) suggests that gravity wave belt difference is associated with <100km waves
% What wavelengths are suggested based on AIRS data


\chapter{Outlook}

- full 3D simulations (cite zhang.. 2005 together with Alexander/Durran...) extend those simulations to upper stratosphere.

- further LIDAR analysis